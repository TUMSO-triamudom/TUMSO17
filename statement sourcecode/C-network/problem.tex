\documentclass[11pt,a4paper]{article}

\newcommand{\tumsoTime}{09:00 น. - 12:00 น.}
\newcommand{\tumsoRound}{1}

\usepackage{../tumso}

\begin{document}

\begin{problem}{เครือข่ายข้อมูล}{standard input}{standard output}{0.5 seconds}{128 megabytes}{140}

เนื่องจากฝ่ายสารสนเทศของโรงเรียนแห่งหนึ่งย่านพญาไท ได้รับคำสั่งจากผู้อำนวยการท่านใหม่ให้วางโครงสร้างเครือข่ายข้อมูลภายในโรงเรียนใหม่ทั้งหมดเนื่องจากอะไรก็ไม่รู้ไม่มีใครกล้าถาม ทางฝ่ายสารสนเทศจึงจัดการประชุมหาวิธีการวางเครือข่ายให้มีประสิทธิภาพมากที่สุด จนได้ข้อสรุปว่าการวางเครื่องแม่ข่ายและเชื่อมกันให้เป็นแบบต้นไม้เป็นวิธีที่ดีที่สุด ทางฝ่ายจึงเริ่มร่างต้นไม้ $N$ ปม จำลองโครงสร้างเครือข่ายข้อมูลโดยให้แต่ละปมของต้นไม้แทนเครื่องเครือข่าย $1$ เครื่อง กำหนดให้ปมที่ $S$ เป็นปมแรกที่เก็บข้อมูลไว้และพร้อมที่จะกระจายข้อมูลให้ปมลูก ซึ่งในเวลา $1$ หน่วยนั้น ปมที่มีข้อมูลอยู่แล้วจะสามารถเลือกส่งต่อข้อมูลให้ปมลูกได้เพียงแค่ปมเดียวเท่านั้น แต่ถ้าในเวลานั้นมีหลายปมที่มีข้อมูลอยู่แล้ว ทุกๆปมที่มีข้อมูลอยู่แล้วสามารถเลือกส่งข้อมูลให้ปมลูกหนึ่งปมในเวลาเดียวกันได้ เมื่อทางฝ่ายได้จัดการวางเครื่องแม่ข่ายตามร่างเป็นที่เรียบร้อยแล้ว ก็ต้องการทราบเวลาที่น้อยที่สุดที่ข้อมูลจะถูกกระจายไปครบทุกปม แต่แค่งานหลักของฝ่ายสารสนเทศก็เยอะเกินพออยู่แล้ว พวกเขาจึงได้ไหว้วานให้ผู้เข้าแข่งขันครั้งนี้ช่วยเหลือโดยมี $140$ คะแนนเป็นสิ่งตอบแทน ดังนั้นงานของท่านคือจงหาเวลาที่น้อยที่สุดที่ทุกปมจะมีข้อมูล เพื่อที่จะได้ตั้งค่าระบบให้มีประสิทธิภาพมากที่สุด 

\InputFile
บรรทัดที่ 1 ประกอบด้วยจำนวนเต็ม $N$ และ $S$ แสดงจำนวนปมในต้นไม้และปมแรกที่เก็บข้อมูลไว้ ($1 \le N \le 10^5$ , $0 \le S \le N-1$)

บรรทัดที่ 2 จนถึงบรรทัดที่ $N$ เป็นข้อมูลเส้นเชื่อมประกอบด้วยจำนวนเต็ม $x_i$ และ $y_i$ แสดงว่าเส้นเชื่อมที่ $i$ เชื่อมระหว่างปม $x_i$ และ $y_i$ โดยที่ $0 \le x_i , y_i \le N-1$

\OutputFile
จำนวนเต็มหนึ่งค่าแสดงเวลาที่น้อยที่สุดที่ทุกปมในต้นไม้จะมีข้อมูลครบทุกปม

\Scoring
ชุดทดสอบจะถูกแบ่งเป็น 3 ชุด จะได้คะแนนในแต่ละชุดก็ต่อเมื่อโปรแกรมให้ผลลัพธ์ถูกต้องในชุดทดสอบย่อยทั้งหมด

\begin{description}

\item[ชุดที่ 1 (16 คะแนน)] จะมี $N=2$

\item[ชุดที่ 2 (36 คะแนน)] รับประกันว่าเป็นต้นไม้เส้นตรง

\item[ชุดที่ 3 (88 คะแนน)] ไม่มีเงื่อนไขเพิ่มเติมจากโจทย์

\end{description}


\Examples

\begin{example}
\exmpfile{example.01}{example.01.a}%
\exmpfile{example.02}{example.02.a}%
\end{example}

\end{problem}

\end{document}