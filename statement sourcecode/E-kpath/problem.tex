\documentclass[11pt,a4paper]{article}

\newcommand{\tumsoTime}{09:00 น. - 12:00 น.}
\newcommand{\tumsoRound}{1}

\usepackage{../tumso}

\begin{document}

\begin{problem}{K-Path}{standard input}{standard output}{1 second}{256 megabytes}{240}

กำหนดต้นไม้ต้นหนึ่งที่มี $n$ จุดยอด โดยที่แต่ละเส้นเชื่อมมีน้ำหนักเป็นจำนวนเต็มบวก เราสามารถที่จะลดหรือเพิ่มน้ำหนักของเส้นเชื่อมหลายๆเส้นเชื่อมได้โดยที่น้ำหนักหลังการเปลี่ยนแปลงห้ามเป็นลบ กำหนด $C(e) = \lvert w_2 - w_1 \rvert$ โดยที่ $w_2$ คือน้ำหนักหลังการเปลี่ยนแปลงน้ำหนักของเส้นเชื่อม $e$ และ $w_1$ คือน้ำหนักก่อนการเปลี่ยนแปลงน้ำหนักของเส้นเชื่อม $e$ จงหาว่า $\sum\limits_{e} C(e)$ สามารถมีค่าได้ต่ำสุดเท่าใดหลังการเปลี่ยนแปลงน้ำหนักของบางเส้นเชื่อมในต้นไม้จึงจะมี path ความยาวเท่ากับ $k$ พอดี

\InputFile
ในบรรทัดแรกมีจำนวนเต็ม $n$ และ $k$ $(1 \leq n \leq 100\,000, 1 \leq k \leq 10^9)$ ซึ่งกำหนดจำนวนจุดยอดในต้นไม้และความยาว path ที่ต้องการ

อีก $n-1$ บรรทัดต่อมาจำมีจำนวนเต็ม $u$ $v$ และ $w$ $(1 \leq u, v \leq n, 1 \leq w \leq 10^9)$ ซึ่งกำหนดว่ามีเส้นเชื่อมจุดยอด $u$ และ $v$ โดยมีความยาวเท่ากับ $w$

\OutputFile
แสดงจำนวนเต็มหนึ่งจำนวนคือค่า $\sum\limits_{e} C(e)$ ที่น้อยที่สุดที่เป็นไปได้


\Scoring
ชุดทดสอบจะถูกแบ่งเป็น 4 ชุด จะได้คะแนนในแต่ละชุดก็ต่อเมื่อโปรแกรมให้ผลลัพธ์ถูกต้องในชุดทดสอบย่อยทั้งหมด
\begin{description}
\item[ชุดที่ 1 (37 คะแนน)] จะมี $n \leq 100$, $k \leq 100$ และมีเส้นเชื่อมจากจุดยอด $u$ ไป $u+1$ สำหรับ $1 \leq u < n$

\item[ชุดที่ 2 (41 คะแนน)] จะมี $n \leq 1\,000$, $k \leq 10^9$

\item[ชุดที่ 3 (43 คะแนน)] จะมี $k \leq 100$

\item[ชุดที่ 4 (119 คะแนน)] ไม่มีเงื่อนไขเพิ่มเติมจากโจทย์
\end{description}

\Examples

\begin{example}
\exmpfile{example.01}{example.01.a}%
\exmpfile{example.02}{example.02.a}%
\end{example}

\end{problem}

\end{document}