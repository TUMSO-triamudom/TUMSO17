\documentclass[11pt,a4paper]{article}

\newcommand{\tumsoTime}{13:00 น. - 16:00 น.}
\newcommand{\tumsoRound}{2}

\usepackage{../tumso}

\begin{document}

\begin{problem}{แฟกทอเรียล}{}{}{1 second}{256 megabytes}{300}

นิยามให้ตัวหารของจำนวนเต็มบวก $x$ หมายถึง จำนวนเต็มบวกทั้งหมดที่นำมาหาร $x$ ได้ลงตัว เช่น ตัวหารของ $20$ ได้แก่ $1, 2, 4, 5, 10, 20$

กำหนดให้ใช้สัญลักษณ์ $\sigma_0(x)$ แทนจำนวนตัวหารของ $x$ เช่น $\sigma_0(20) = 6$

กำหนดจำนวนเต็ม $N$ และ $K$ ให้ จงนับจำนวนเต็มบวก $x$ ทั้งหมดที่ตรงตามเงื่อนไข
\begin{itemize}
\item $x$ เป็นตัวหารของ $N!$ ($x|N!$)
\item $x$ มีตัวหารทั้งหมด $K$ ตัวพอดี ($\sigma_0(x) = K$)
\end{itemize}
รับประกันว่าค่า $K$ ที่กำหนดให้จะมี $\sigma_0(K) \leq 4$

\InputFile

ข้อมูลนำเข้ามีทั้งหมด $T+1$ บรรทัด

บรรทัดแรก จำนวนเต็ม $T$ $(1 \le T \le 50)$ จำนวนชุดทดสอบ

อีก $T$ บรรทัดระบุ $N$ และ $K$ $(1 \le N \le 10^6, 1 \le K \le 10^9)$

\OutputFile
มี $T$ บรรทัด เป็นคำตอบของคำถาม

\Scoring
ชุดทดสอบจะถูกแบ่งเป็น 3 ชุด จะได้คะแนนในแต่ละชุดก็ต่อเมื่อโปรแกรมให้ผลลัพธ์ถูกต้องในชุดทดสอบย่อยทั้งหมด

\begin{description}
\item[ชุดที่ 1 (59 คะแนน)] จะมี $1 \le N \le 8, 1 \le K \leq 10^2$
\item[ชุดที่ 2 (82 คะแนน)] จะมี $1 \le N \le 10^2, 1 \le K \leq 10^4$
\item[ชุดที่ 3 (159 คะแนน)] ไม่มีเงื่อนไขเพิ่มเติมจากโจทย์
\end{description}

\Examples

\begin{example}
\exmpfile{example.01}{example.01.a}%
\exmpfile{example.02}{example.02.a}%
\end{example}

\Note

ในตัวอย่างที่ $1$

เมื่อ $N=8$, $K=2$ จะมีจำนวนเต็มบวก $x$ ที่ตรงตามเงื่อนไขทั้งหมด $4$ ตัว ได้แก่ $2, 3, 5, 7$

เมื่อ $N=8$, $K=3$ จะมีจำนวนเต็มบวก $x$ ที่ตรงตามเงื่อนไขทั้งหมด $4$ ตัว ได้แก่ $4, 9$

เมื่อ $N=8$, $K=4$ จะมีจำนวนเต็มบวก $x$ ที่ตรงตามเงื่อนไขทั้งหมด $7$ ตัว ได้แก่ $6, 8, 10, 14, 15, 21, 35$

เมื่อ $N=8$, $K=5$ จะมีจำนวนเต็มบวก $x$ ที่ตรงตามเงื่อนไขทั้งหมด $1$ ตัว ได้แก่ $16$

เมื่อ $N=8$, $K=6$ จะมีจำนวนเต็มบวก $x$ ที่ตรงตามเงื่อนไขทั้งหมด $7$ ตัว ได้แก่ $12, 18, 20, 28, 32, 45, 63$

\end{problem}

\end{document}