\documentclass[11pt,a4paper]{article}

\newcommand{\tumsoTime}{13:00 น. - 16:00 น.}
\newcommand{\tumsoRound}{2}

\usepackage{../tumso}

\begin{document}

\begin{problem}{Macrosoft Doors}{standard input}{standard output}{1 second}{256 megabytes}{100}

ยินดีด้วย คุณถูกคัดเลือกให้เข้าไปทำงานใน Macrosoft ได้เพราะมีคนเห็นแวว เห็นศักยภาพ เห็นความสามารถในตัวคุณตอนที่คุณเข้าค่ายโอลิมปิกวิชาการ

หน้าที่ของคุณคือ ออกแบบระบบการอัพเดทของระบบปฏิบัติการ Macrosoft Doors โดยการอัพเดทจะมีขั้นตอนดังนี้

\begin{enumerate}
\item แจ้งเตือนผู้ใช้ว่าจะทำการอัพเดท Macrosoft Doors
\item บังคับ restart ระบบ ให้ผู้ใช้อัพเดท Macrosoft Doors ทันที
\end{enumerate}

คุณสามารถคาดการณ์ได้ว่าผู้ใช้ Macrosoft Doors จะต้องทำงานเป็นเวลา $N$ นาที โดย
\begin{itemize}
\item ในกรณีปกติ ผู้ใช้จะทำงานในนาทีที่ $i$ ($1 \leq i \leq N$) ได้ประสิทธิภาพ $A_i$ หน่วย
\item หลังการแจ้งเตือนว่าจะอัพเดท ประสิทธิภาพในนาทีที่ $i$ จะกลายเป็น $B_i$ หน่วย
\item หลังการอัพเดทแล้ว ประสิทธิภาพในนาทีที่ $i$ จะกลายเป็น $C_i$ หน่วย
\end{itemize}

คุณจะต้องเลือกเวลามา 2 เวลา โดยเวลาที่เลือกต้องเป็นช่วงต่อระหว่างนาทีที่ผู้ใช้กำลังทำงานอยู่เท่านั้น (เลือกเวลาก่อนเริ่มทำงานหรือหลังทำงานเสร็จไม่ได้) เพื่อทำการแจ้งเตือนอัพเดท แล้วหลังจากนั้นอีกอย่างน้อย 1 นาที จึงจะบังคับ restart เพื่ออัพเดท Macrosoft Doors

เป้าหมายของคุณคือแจ้งเตือนแล้วอัพเดทยังไงก็ได้ให้ผู้ใช้\textbf{ทำงานได้ประสิทธิภาพรวมน้อยที่สุด} จงเขียนโปรแกรมเพื่อหาประสิทธิภาพรวมที่น้อยที่สุดที่เป็นไปได้

\begin{center}
\includegraphics[width=12cm]{trapcard.jpg}
\end{center}

\InputFile
ข้อมูลนำเข้ามีทั้งหมด $4$ บรรทัด

บรรทัดแรก ประกอบด้วยจำนวนเต็ม $N$ ($3 \leq N \leq 100\,000$)

บรรทัดที่ $2$ ประกอบด้วยจำนวนเต็ม $A_1, A_2, \dots, A_N$ ($1 \leq A_i \leq 10^9$)

บรรทัดที่ $3$ ประกอบด้วยจำนวนเต็ม $B_1, B_2, \dots, B_N$ ($1 \leq B_i \leq 10^9$)

บรรทัดที่ $4$ ประกอบด้วยจำนวนเต็ม $C_1, C_2, \dots, C_N$ ($1 \leq C_i \leq 10^9$)

\OutputFile
ให้ตอบประสิทธิภาพการทำงานรวมที่น้อยที่สุดที่เป็นไปได้

\Scoring
ชุดทดสอบจะถูกแบ่งเป็น 3 ชุด จะได้คะแนนในแต่ละชุดก็ต่อเมื่อโปรแกรมให้ผลลัพธ์ถูกต้องในชุดทดสอบย่อยทั้งหมด

\begin{description}

\item[ชุดที่ 1 (15 คะแนน)] จะมี $N \leq 500$

\item[ชุดที่ 2 (22 คะแนน)] จะมี $N \leq 2\,000$

\item[ชุดที่ 3 (63 คะแนน)] ไม่มีเงื่อนไขเพิ่มเติม

\end{description}

\Examples

\begin{example}
\exmp{3
1 2 3
1 2 3
1 2 3
}{6}%
\exmp{5
6 2 4 5 9
2 3 5 6 8
3 2 5 4 1
}{18}%
\end{example}

\end{problem}

\end{document}