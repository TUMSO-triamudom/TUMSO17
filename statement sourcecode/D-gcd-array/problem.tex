\documentclass[11pt,a4paper]{article}

\newcommand{\tumsoTime}{09:00 น. - 12:00 น.}
\newcommand{\tumsoRound}{1}

\usepackage{../tumso}

\begin{document}

\begin{problem}{น้องออโต้กับตัวหารร่วมมาก}{standard input}{standard output}{1 second}{256 megabytes}{180}

มีเด็กชายคนหนึ่งชื่อออโต้ ได้รับชุดตัวเลข $A$ ที่ประกอบด้วยตัวเลขทั้งหมด $N$ ตัว มาจากอาจารย์ท่านหนึ่ง ให้ $a_i$  แทนเลขตัวที่ $i$ ในชุดตัวเลข $A$
ให้ $[l, r]$ แทนช่วงที่ติดกันของตัวเลขตั้งแต่ตัวที่ $l$ ถึงตัวที่ $r$ หรือ $a_l, \dots, a_r$

ออโต้เป็นเด็กที่สนใจในวิชาคณิตศาสตร์มากๆ และเป็นคนที่หลงใหลใน "ตัวหารร่วมมาก" นอกจากนี้ เขายังได้รับจำนวนเต็มบวก $K$ มาอีกตัวหนึ่งด้วย

ออโต้อยากจะรู้ว่า ในชุดตัวเลขที่ได้รับมานั้น จะมีคู่ $[l, r]$ ที่ทำให้ตัวหารร่วมมากของตัวเลขใน $[l, r]$ นั้น มีค่าไม่เกิน $K$

เนื่องจากชุดตัวเลขนี้ยาวมาก ออโต้จึงอยากให้คุณเขียนโปรแกรมในการนับจำนวนคู่ $[l, r]$ ที่เป็นไปได้ทั้งหมด

\InputFile
ข้อมูลนำเข้ามีทั้งหมด $2$ บรรทัด

บรรทัดแรก จำนวนเต็ม $N$ ($1 \le N \le 2\cdot10^5$) และจำนวนเต็ม $K$ ($1 \le K \le 3\cdot10^3$)

บรรทัดที่สอง จำนวนเต็มทั้งหมด $N$ จำนวน ได้แก่ $a_1, a_2, \dots, a_N$ ($1 \le a_i \le 3\cdot10^3$)

\OutputFile
ตอบจำนวนเต็มเพียงหนึ่งตัว จำนวนคู่ $[l, r]$ ทั้งหมดที่เป็นไปได้ดังที่โจทย์ได้กล่าวไว้

\Scoring
ชุดทดสอบจะถูกแบ่งเป็น 3 ชุด จะได้คะแนนในแต่ละชุดก็ต่อเมื่อโปรแกรมให้ผลลัพธ์ถูกต้องในชุดทดสอบย่อยทั้งหมด

\begin{description}
\item[ชุดที่ 1 (35 คะแนน)] จะมี $1 \le N \le 1000$

\item[ชุดที่ 2 (82 คะแนน)] จะมี $1 \le N \le 5\cdot10^4$

\item[ชุดที่ 3 (63 คะแนน)] ไม่มีเงื่อนไขเพิ่มเติม
\end{description}

\Examples

\begin{example}
\exmp{3 2
2 2 4
}{5
}%
\exmp{4 2
1 2 4 8
}{7
}%
\end{example}

\end{problem}

\end{document}
